\documentclass[12pt,twoside]{report}

\usepackage[utf8]{inputenc}
\usepackage[spanish]{babel}

\usepackage{amsmath}
\usepackage{amsfonts}
\usepackage{amssymb}

\usepackage{setspace}

\usepackage{enumitem}

\usepackage[normalem]{ulem}

\usepackage[table,xcdraw]{xcolor}

\usepackage[export]{adjustbox}

%%%%%%%%%%%%%%%%%%%%%%%%%%%%%%%%%%%%%%%%%%%%%%%%%%%%%%%%%%%%%%%%%%%%%%%%%%%%%

% Definitions for the title page
\newcommand{\reporttitle}{
	Solicitud de Subvención\\
	y \\
	Memoria de Actividades 2014-15
}
\newcommand{\reportauthor}{Asociación Club de Robótica-Mecatrónica}



\setlength{\parskip}{\baselineskip}
\setlength{\parindent}{0pt}

\usepackage{hyperref}
\hypersetup{
    hyperindex, linktoc=page, breaklinks,
    pdfauthor=\reportauthor,pdftitle=\reporttitle
}
%%%%%%%%%%%%%%%%%%%%%%%%%%%%%%%%%%%%%%%%%%%%%%%%%%%%%%%%%%%%%%%%%%%%%%%%%%%%%

% load some definitions and default packages
%%%%%%%%%%%%%%%%%%%%%%%%%%%%%%%%%%%%%%%%%
% University Assignment Title Page 
% LaTeX Template
% Version 1.0 (27/12/12)
%
% This template has been downloaded from:
% http://www.LaTeXTemplates.com
%
% Original author:
% WikiBooks (http://en.wikibooks.org/wiki/LaTeX/Title_Creation)
%
% License:
% CC BY-NC-SA 3.0 (http://creativecommons.org/licenses/by-nc-sa/3.0/)
% 
%
%%%%%%%%%%%%%%%%%%%%%%%%%%%%%%%%%%%%%%%%%
%----------------------------------------------------------------------------------------
%	PACKAGES AND OTHER DOCUMENT CONFIGURATIONS
%----------------------------------------------------------------------------------------
%\usepackage[a4paper,left=2.8cm,right=2.8cm,vmargin=2.0cm,includeheadfoot]{geometry}
\usepackage[a4paper,left=3.5cm,right=2.1cm,vmargin=2.0cm,includeheadfoot]{geometry}

\usepackage{textpos}

\usepackage{tabularx,longtable,multirow,subfigure,caption}
\usepackage{fncylab} %formatting of labels
\usepackage{fancyhdr} % page layout
\usepackage{url} % URLs

\usepackage{amsmath}
\usepackage{graphicx}
\usepackage{dsfont}

\usepackage{array}
\usepackage{latexsym}



%%% Default fonts
\renewcommand*{\rmdefault}{bch}
\renewcommand*{\ttdefault}{cmtt}



%%% Default settings (page layout)
\setlength{\parindent}{0em}  % indentation of paragraph

\setlength{\headheight}{14.5pt}
\pagestyle{fancy}
\renewcommand{\chaptermark}[1]{\markboth{\chaptername\ \thechapter.\ #1}{}} 

\fancyfoot[EL,OR]{\sffamily\textbf{\thepage}}%Page no. in the left on odd pages and on right on even pages
\fancyfoot[OC,EC]{\sffamily }
\renewcommand{\headrulewidth}{0.1pt}
\renewcommand{\footrulewidth}{0.1pt}
\captionsetup{margin=10pt,font=small,labelfont=bf}


%--- chapter heading

\def\@makechapterhead#1{%
  \vspace*{10\p@}%
  {\parindent \z@ \raggedright \sffamily
    \interlinepenalty\@M
    \Huge\bfseries \thechapter \space\space #1\par\nobreak
    \vskip 30\p@
  }}

%---chapter heading for \chapter*  
\def\@makeschapterhead#1{%
  \vspace*{10\p@}%
  {\parindent \z@ \raggedright
    \sffamily
    \interlinepenalty\@M
    \Huge \bfseries  #1\par\nobreak
    \vskip 30\p@
  }}

\allowdisplaybreaks


\date{Diciembre de 2015}

\addto\captionsspanish{\renewcommand{\chaptername}{Parte}}

\begin{document}

\begin{titlepage}

\newcommand{\HRule}{\rule{\linewidth}{1mm}} % Defines a new command for the horizontal lines, change thickness here


%----------------------------------------------------------------------------------------
%	LOGO SECTION
%----------------------------------------------------------------------------------------

\includegraphics[width = 6cm]{fotos/logo-eps.png}
\hfill
\includegraphics[width = 6cm]{fotos/logo-uam.png}



\center % Center remainder of the page

%----------------------------------------------------------------------------------------
%	HEADING SECTIONS
%----------------------------------------------------------------------------------------

%\textsc{\Large Escuela Politécnica Superior}\\[0.1cm] 
%\textsc{\Large Universidad Autónoma de Madrid}\\[0.5cm] 
\vspace{1cm}

%----------------------------------------------------------------------------------------
%	TITLE SECTION
%----------------------------------------------------------------------------------------


\HRule \\[0.4cm]
\begin{spacing}{1.5}
{ \fontsize{0.8cm}{1em} \bfseries \reporttitle}\\ % Title of your document
\vspace{0.5cm}
{ \fontsize{0.7cm}{1em} \bfseries \reportauthor} \\
\end{spacing}
\HRule \\[1.5cm]


\includegraphics[width = 7cm]{fotos/logo_crm-192x192.png}

Asociación Club de Robótica-Mecatrónica \\
Local B-111 -- Escuela Politécnica Superior

\vfill

\textsc{\Large Universidad Autónoma de Madrid}\\[0.5cm] 

\vfill

%----------------------------------------------------------------------------------------
%	FOOTER & DATE SECTION
%----------------------------------------------------------------------------------------


\makeatletter
{ \Large \@date }
\vfill
\makeatother


\end{titlepage}

\clearpage{\pagestyle{empty}\cleardoublepage}


% page numbering etc.
\pagenumbering{roman}
\setcounter{page}{1}
\pagestyle{fancy}






\begin{spacing}{0.1}
\tableofcontents 
\end{spacing}

\clearpage{\pagestyle{empty}\cleardoublepage}

\pagenumbering{arabic}
\setcounter{page}{1}

\fancyhead[LE,RO]{\slshape}
%\fancyhead[LE,RO]{\slshape \rightmark}
\fancyhead[LO,RE]{\slshape \leftmark}

%%%%%%%%%%%%%%%%%%%%%%%%%%%%%%%%%%%%


\chapter{Presupuesto para nuevas actividades}

\section{Solicitud de subvención}







\chapter{Memoria del curso anterior}

\section{Talleres y proyectos internos}

\subsection{Construcción de un cuadrucóptero}

\subsection{Uso de la impresora 3D por estudiantes}

\subsection{Re-organización del local para fomentar la participación}

\begin{itemize}
\item Actualización de la página web
\item Limpieza del local (reciclado de equipos obsoletos que ocupaban espacio, mesas despejadas para facilitar la labor del equipo de limpieza)
\item Organización del material de los armarios y de las herramientas gracias a un panel de madera con ganchos.
\item Cada estudiante puede solicitar una caja de proyecto donde guardar todo el material que necesite. Dichas cajas están etiquetadas con su nombre y año, de este modo es posible organizar mejor el inventario disponible.
\end{itemize}

El nuevo enfoque del Club de Robótica es apoyar a cualquier miembro de la comunidad universitaria que quiera llevar a cabo proyectos relacionados con la robótica. Es decir, tanto estudiantes como profesores pueden inscribirse y así disponer de un espacio de trabajo agradable con herramientas de uso común (impresoras 3D, soldadores, sierras, alicates, destornilladores, etc) así como los materiales necesarios (cables, componentes, motores, baterías, etc).

Además disponemos de un foro donde nos ayudamos unos a otros, y periódicamente seguimos organizando actividades para fomentar la robótica entre los estudiantes. 


\section{Participación en eventos nacionales}


\subsection{Concurso de resolución de laberintos en la OSHWDem 2015}


\subsection{V jornada GMV de robótica}




\chapter{Junta directiva actualizada}




\end{document}
