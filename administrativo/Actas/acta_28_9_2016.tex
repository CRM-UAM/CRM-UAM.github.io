\documentclass[a4paper]{article}

%% Language and font encodings
\usepackage[spanish]{babel}
\usepackage[utf8x]{inputenc}
\usepackage[T1]{fontenc}

%% Sets page size and margins
\usepackage[a4paper,top=3cm,bottom=2cm,left=3cm,right=3cm,marginparwidth=1.75cm]{geometry}

%% Useful packages
\usepackage{amsmath}
\usepackage{graphicx}
\usepackage[colorinlistoftodos]{todonotes}
\usepackage[colorlinks=true, allcolors=blue]{hyperref}

\title{Club de Robótica Mecatrónica}
\author{Universidad Autónoma de Madrid}

\begin{document}

El \textbf{28 de septiembre} a las 13:10 tuvo lugar la \textbf{Asamblea anual ordinaria del Club de Robótica-Mecatrónica} en el local del club. Asisten:

\begin{itemize}
\item Carlos García Saura (Presidente)
\item Pablo Molins Ruano (Vocal)
\item Carlos González Sacristán
\item Fernando Álvarez Jimenénez
\item Juan José Rodriguez Díaz
\item Rafael Leira Osuna
\item Eduardo Hilario Serrano
\item Santiago Valderrabano Zamorano
\item Raquel de Benito
\item Lázaro López Ibáñez
\item Alfredo Sanz Ortega
\item Inés Lozano Jiménez
\item Marina Alonso Cortés
\item Rafael Sanchéz Sanchéz (?)
\end{itemize}

Excusan su ausencia:
\begin{itemize}
\item Rodrigo Jiménez (Vicepresidente)
\item Cristina Kasner Tourné (Secretaria)
\item Jaime Aragón (Tesorero)
\item Guillermo Guridi Mateos
\end{itemize}


\section{Presentación y bienvenida}

El Presidente inicia la reunión haciendo una presentación a los nuevos miembros que se incorporan este año. Muestra la página web de la asocación \url{http://crm.ii.uam.es/} explicando cada una de sus secciones. Se muestra también las listas de correo y se explica que se utilizan para enviar información relevante sobre las actividades, talleres y reuniones del club además de usarse como foro de discusión.

Se hace una ronda de presntación, para que nuevos y antiguos miembros se vayan conociendo.

\section{Repaso de los proyectos que se ha realizado durante este curso (de cara a empezar a escribir la memoria).}

El Presidente explica que cada año hay que presentar ante rectorado una memoria. Explica en qué consiste dicha memoria, usando como ejemplo la memoria del año anterior (disponible en \url{http://crm-uam.github.io/administrativo/Memorias/2015/memoria.pdf}).

El presidente enumera de los proyectos que se han realizado este año (y que, por lo tanto, deberán aparecer en la memoria):

\begin{enumerate}
\item Se ha continuado con el desarrollo del dron, proyecto empezado en el curso 2015-2016. Repo: \url{https://github.com/CRM-UAM/cuadricoptero}
\item Participación en competiciones de robótica. Se han desarrollado robots sigue líneas y para resolver laberintos \url{https://github.com/CRM-UAM/CRMaze}. Se participó en la OSHWDem en noviembre de 2015 en A Coruña y la Robolid del 6-8 de abril de 2016 en Valladolid.
\item Coche teleridigido. Repo: \url{https://github.com/CRM-UAM/coche-RC}
\item Automatismos del local. Tuberias musicales. Repo: \url{https://github.com/CRM-UAM/automatismos-del-local}
\item Phogo. Proyecto de robótica educativa. Repo: \url{https://github.com/CRM-UAM/Phogo}
\end{enumerate}

\section{Presentación de proyectos que se están realizando actualmente en el CRM.}

Se presentan los proyectos que actualmente se están desarrollando en el club, animando a que los miembros interesados contacten con la persona organizadora de cada proyecto para recibir más información y pensar cómo podría participar:

\begin{enumerate}
\item Escaner 3D. Hablar con Eduardo Hilario.
\item Modernizar coches. Hablar con Lázaro López.
\item Microcontroladores. Hablar con Eudardo Hilario.
\item FPGAs libres. Hablar con Carlos García o Pablo Molins.
\item Phogo. Hablar con Carlos García, Carlos González o Pablo Molins.
\item Coche teledirigido. Hablar con Carlos González.
\item Dron. Hablar con Carlos García, Jaime Aragón o Rodrigo Jiménez.
\item Robots de competición. Hablar con Carlos García.
\end{enumerate}

\section{Repaso de la distribución de tareas de los cargos de la nueva junta.}

Miembros de la actual junta (Carlos G.S., Cristina K.T. y Pablo M.R.) presentan una propuesta para modificar las atribuciones de cada miembro de la junta directiva del club. Las atribuciones actualmente recogidas en los estatutos no se siguen y de facto el presidente carga con todas, con ayuda esporádica de algunos otros miembros. 

La nueva propuesta introduce una nueva figura, la vocalía de actividades, manteniendo el resto de figuras: presidencia, vicepresidencia, tesorería, secretaría y vocalías. El documento completo de la propuesta se puede consultar en \url{http://crm-uam.github.io/administrativo/Propuesta\%20cargos\%20junta.pdf}.

Se presenta la propuesta, se discute y finalmente se aprueba.

\section{Elección de la nueva junta.}

Tras haber discutido las nuevas atribuciones, se decide elegir una nueva junta que ya siga el nuevo formato. Finalmente, la nueva junta queda así:

\begin{itemize}
\item Presidente: Pablo Molins Ruano.
\item Vicepresidenta: Cristina Kasner Tourné.
\item Tesorero: Eduardo Hilario Serrano.
\item Secretario: Lázaro López Ibáñez.
\item Vocal de Actividades: Alfredo Sanz Ortega.
\item Vocales: Carlos García Saura, Carlos Gonzalez Sacristán, Rafael Leira Osuna.
\end{itemize}

\section{Estudiar nuevos proyectos para este curso.}

Para no alargar más reunión, y para facilitar que todos los miembros puedan participar en el debate, aún no habiendo asistido, se decide discutir los nuevos proyectos por la lista de correo del club.

\section{Ruegos y preguntas.}

Eduardo Hilario plantea la posibilidad de buscar un nuevo local, ya que el actual se empieza a quedar pequeño con la llegada de nuevos miembros. Se habla de que ahora podría ser un buen momento, porque todavía hay espacio libre en el edificio C que a lo mejor podrían asignarnos. Se discute si el edificio C sería buena localización, al alejarnos aún más del edificio A, donde los alumnos pasan la mayor parte de su tiempo, aunque el edificio C tiene más alumnos (al haber algunos laboratorios y aulas docentes) que el edificio B, así que a lo mejor haría más visible al club que en el sótano del B. Se acuerda preguntar al subdirector correspondiente y ver qué nos ofrecen, y en función de eso decidir si se hace la mudanza.

Carlos García plantea el tema de ir a la OSHWDem de este año en A Coruña. Se debate el tema de coste, y si sería posible contar con más fondos de los que aún quedan al club para este año para pagar parte del coste del viaje a quién asista. Se queda Carlos encargado de preguntar y organizar el tema.

Finalmente, \textbf{se cierra la sesión a las 14:45.}

\end{document}
